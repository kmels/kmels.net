\chapter*{Introducci\'{o}n}

En el a\~{n}o 2007, la Ciudad de Guatemala ten\'{i}a una poblaci\'{o}n estimada de m\'{a}s de 3 millones de personas, el departamento con m\'{a}s poblaci\'{o}n de Guatemala  \cite{WolframAlpha-PoblacionGuatemala2007}. La necesidad de manejar el manejo de informaci\'{o}n relacionada con cada persona como por ejemplo su registro civil, la cantidad de veces que se ha casado y divorciado o incluso cual es la educaci\'{o}n m\'{a}xima que ha recibido, nos lleva a proponer un Sistema de Informaci\'{o}n que se adec\'{u}e muy bien a la administraci\'{o}n del mismo .

El presente trabajo es un an\'{a}lasis completo de un sistema a realizar para el registro nacional de las personas de la ciudad de Guatemala. Este sistema almacena y maneja los registros y actos civiles de la vida de cada ciudadano: su nacimiento, nombres de la madre y padre, fecha y lugar de nacimiento, tipo de parto, etc\'{e}tera. 

El sistema tambi\'{e}n tiene la capacidad de almacenar informaci\'{o}n relacionada a su estado civil: datos de matrimonio en caso que aplique: informaci\'{o}n del c\'{o}nyugue, lugar de origen, fecha del matrimonio, lugar donde se realiz\'{o}, quien los cas\'{o} 