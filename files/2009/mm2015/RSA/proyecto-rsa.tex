\documentclass{article}
% Change "article" to "report" to get rid of page number on title page
\usepackage{amsmath,amsfonts,amsthm,amssymb}
\usepackage{setspace}
\usepackage{Tabbing}
\usepackage{fancyhdr}
\usepackage{lastpage}
\usepackage{extramarks}
\usepackage{chngpage}
\usepackage{soul,color}
\usepackage{sagetex}
\usepackage{graphicx,float,wrapfig}

% In case you need to adjust margins:
\topmargin=-0.45in      %
\evensidemargin=0in     %
\oddsidemargin=0in      %
\textwidth=6.5in        %
\textheight=9.0in       %
\headsep=0.25in         %

% Homework Specific Information
\newcommand{\hmwkTitle}{Proyecto RSA}
\newcommand{\hmwkDueDate}{Jueves,\ Nov\ 11,\ 2009}
\newcommand{\hmwkClass}{MM2015}
\newcommand{\hmwkClassTime}{}
\newcommand{\hmwkClassInstructor}{H. Villafuerte}
\newcommand{\hmwkAuthorName}{Carlos E. L\'{o}pez Camey}

% Setup the header and footer
\pagestyle{fancy}                                                       %
\lhead{\hmwkAuthorName}                                                 %
\chead{\hmwkClass\ (\hmwkClassInstructor\ \hmwkClassTime): \hmwkTitle}  %
\rhead{\firstxmark}                                                     %
\lfoot{\lastxmark}                                                      %
\cfoot{}                                                                %
\rfoot{Page\ \thepage\ of\ \pageref{LastPage}}                          %
\renewcommand\headrulewidth{0.4pt}                                      %
\renewcommand\footrulewidth{0.4pt}                                      %

% This is used to trace down (pin point) problems
% in latexing a document:
%\tracingall

%%%%%%%%%%%%%%%%%%%%%%%%%%%%%%%%%%%%%%%%%%%%%%%%%%%%%%%%%%%%%
% Some tools
%\newcommand{\enterProblemHeader}[1]{\nobreak\extramarks{#1}{#1 continued on next page\ldots}\nobreak%
   %                                 \nobreak\extramarks{#1 (continued)}{#1 continued on next page\ldots}\nobreak}%
\newcommand{\exitProblemHeader}[1]{\nobreak\extramarks{#1 (continued)}{#1 continued on next page\ldots}\nobreak%
                                   \nobreak\extramarks{#1}{}\nobreak}%

\newlength{\labelLength}
\newcommand{\labelAnswer}[2]
  {\settowidth{\labelLength}{#1}%
   \addtolength{\labelLength}{0.25in}%
   \changetext{}{-\labelLength}{}{}{}%
   \noindent\fbox{\begin{minipage}[c]{\columnwidth}#2\end{minipage}}%
   \marginpar{\fbox{#1}}%

   % We put the blank space above in order to make sure this
   % \marginpar gets correctly placed.
   \changetext{}{+\labelLength}{}{}{}}%

\setcounter{secnumdepth}{0}
\newcommand{\homeworkProblemName}{}%
\newcounter{homeworkProblemCounter}%
\newenvironment{homeworkProblem}[1][Problem \arabic{homeworkProblemCounter}]%
  {\stepcounter{homeworkProblemCounter}%
   \renewcommand{\homeworkProblemName}{#1}%
   \section{\homeworkProblemName}%
   %\enterProblemHeader{\homeworkProblemName}
   }%
  %{\exitProblemHeader{\homeworkProblemName}}%

\newcommand{\problemAnswer}[1]
  {\noindent\fbox{\begin{minipage}[c]{\columnwidth}#1\end{minipage}}}%

\newcommand{\problemLAnswer}[1]
  {\labelAnswer{\homeworkProblemName}{#1}}

\newcommand{\homeworkSectionName}{}%
\newlength{\homeworkSectionLabelLength}{}%
\newenvironment{homeworkSection}[1]%
  {% We put this space here to make sure we're not connected to the above.
   % Otherwise the changetext can do funny things to the other margin

   \renewcommand{\homeworkSectionName}{#1}%
   \settowidth{\homeworkSectionLabelLength}{\homeworkSectionName}%
   \addtolength{\homeworkSectionLabelLength}{0.25in}%
   \changetext{}{-\homeworkSectionLabelLength}{}{}{}%
   \subsection{\homeworkSectionName}%
   \enterProblemHeader{\homeworkProblemName\ [\homeworkSectionName]}}%
  {\enterProblemHeader{\homeworkProblemName}%

   % We put the blank space above in order to make sure this margin
   % change doesn't happen too soon (otherwise \sectionAnswer's can
   % get ugly about their \marginpar placement.
   \changetext{}{+\homeworkSectionLabelLength}{}{}{}}%

\newcommand{\sectionAnswer}[1]
  {% We put this space here to make sure we're disconnected from the previous
   % passage

   \noindent\fbox{\begin{minipage}[c]{\columnwidth}#1\end{minipage}}%
   \enterProblemHeader{\homeworkProblemName}\exitProblemHeader{\homeworkProblemName}%
   \marginpar{\fbox{\homeworkSectionName}}%

   % We put the blank space above in order to make sure this
   % \marginpar gets correctly placed.
   }%

%%%%%%%%%%%%%%%%%%%%%%%%%%%%%%%%%%%%%%%%%%%%%%%%%%%%%%%%%%%%%


%%%%%%%%%%%%%%%%%%%%%%%%%%%%%%%%%%%%%%%%%%%%%%%%%%%%%%%%%%%%%
% Make title
\title{\textmd{\textbf{\hmwkClass:\ \hmwkTitle}}\\\normalsize\vspace{0.1in}\small{Para entregar\ el\ \hmwkDueDate}\\\vspace{0.1in}\large{\textit{\hmwkClassInstructor\ \hmwkClassTime}}}
\date{}
\author{\textbf{\hmwkAuthorName}}
%%%%%%%%%%%%%%%%%%%%%%%%%%%%%%%%%%%%%%%%%%%%%%%%%%%%%%%%%%%%%

\begin{document}
\begin{spacing}{1.1}
\maketitle
% Uncomment the \tableofcontents and \newpage lines to get a Contents page
% Uncomment the \setcounter line as well if you do NOT want subsections
%       listed in Contents
%\setcounter{tocdepth}{1}
%\tableofcontents


%\newpage

% When problems are long, it may be desirable to put a \newpage or a
% \clearpage before each homeworkProblem environment
\begin{homeworkProblem}[Fase de experimentaci\'{o}n/Conocimiento b\'{a}sico ]

\subsection{(c) Factorizar 18 y 27 en primos, qu\'{e} podemos decir de gcd(18,27)?}

\begin{sageblock}
sage: factor(18)
2 * 3^2
sage: factor(27)
3^3
\end{sageblock}

Vemos que los dos tienen como factor com\'{u}n el 3, de hecho, $3^2$. Podemos decir que gcd(18,27) = 9 precisamente, porque ese es el mayor divisor que tienen en com\'{u}n.

\begin{sageblock}
sage: gcd(18,27)
9
\end{sageblock}

\subsection{(d) Factorizar 11 y 17 en primos, qu\'{e} podemos decir de gcd(11,17)?}

\begin{sageblock}
sage: factor(11)
11
sage: factor(17)
17
\end{sageblock}

Nos damos cuenta que 11 y 17 no tienen descomposici\'{o}n en factores, es decir que no tienen ning\'{u}n factor en com\'{u}n exceptuando al 1.
\begin{sageblock}
sage: gcd(11,17)
1
\end{sageblock}


\subsection{(f) Factorizar 32 y 45 en primos, qu\'{e} podemos decir de gcd(32,45)?}

\begin{sageblock}
sage: factor(32)
2^5
sage: factor(45)
3^2 * 5
\end{sageblock}

Nos damos cuenta (de nuevo) que 32 y 45 no tienen ning\'{u}n factor en com\'{u}n. En efecto, 
\begin{sageblock}
sage: gcd(32,45)
1
\end{sageblock}



\end{homeworkProblem}






\begin{homeworkProblem}[Generamos nuestra llave p\'{u}blica]
\subsection{Encontremos dos primos $p_1$ y $p_2$}

\begin{sageblock} 
#find primes
p1set = False;  p2set = False
reversedPrimes = list(reversed(list(primes(80)))) #len(str(2^80)) is large enough
i = 0
while not p2set:
    prime = reversedPrimes[i]
    pnumber = 2**prime -1

    if is_prime(pnumber):
        if p1set:
            p2 = pnumber
            print 'retorno'
            p2set = True
        else:
            p1 = pnumber
            p1set = True
    i+=1
\end{sageblock} 

Obtenemos $p_1 = 2305843009213693951$ y $p_2 = 2147483647 \implies n = 4951760154835678088235319297$

\subsection{Encontremos un $e \in \mathbb{N}$ tal que $gcd(e,\phi(n))=1$}

Aqu\'{i}, $\phi(n)$ representa al n\'{u}mero de co-primos de $n$ menores que el mismo tambi\'{e}n llamada funci\'{o}n phi de Euler. En sage, esta est\'{a} implementada como \begin{sageblock}euler_phi(n)\end{sageblock}.

Recordemos que $gcd(p,q) = k$ se refiere al n\'{u}mero $k$ mayor posible que es divisor de $p$ y de $q$ al mismo tiempo, es decir, al m\'{a}ximo com\'{u}n divisor.

\begin{sageblock}
def get_e(phi):
    e = 2
    while gcd(e,phi)!=1:
        e +=1
    return e
\end{sageblock}

Que, obteniendo un (cualquiera) $e$ que cumpla para $phi(n) = 4951760152529835076874141700$, tenemos $e=17$
\subsection{Llave p\'{u}blica}

Nuestra llave p\'{u}blica es entonces, la pareja ordenada $(n,e) = (4951760154835678088235319297, 17)$
\end{homeworkProblem}

\begin{homeworkProblem}[Encontrando nuestra llave privada]

\subsection{Encontrar $d \in \mathbb{Z}$, tal que $d\cdot e = 1 (\mod \phi(n))$ }

\begin{sageblock}
def get_d(e,phi):
    return inverse_mod(e,phi)
\end{sageblock}

Obtenemos entonces, $d=4077920125612805357425763753$ 

\subsection{Llave privada}

Nuestra llave privada es, la tripleta $(p_1,p_2,d)$ = $(2305843009213693951, 2147483647, 4077920125612805357425763753 )$
\end{homeworkProblem}

\begin{homeworkProblem}[Como encriptamos y desencriptamos]

Para un entero $m<n$, encriptar usando $c = m^e (\mod n)$, i.e.

\begin{sageblock}
def get_rsa_number_encryption(m,publicKey):
    e = publicKey[1]
    n = publicKey[0]
    return power_mod(m,e,n)
\end{sageblock}

Desencriptamos $c$ usando $m = c^d (\mod n))$, i.e.

\begin{sageblock}
def get_rsa_number_decryption(c,privateKey):
    n = privateKey[0]*privateKey[1]
    d = privateKey[2]
    return power_mod(c,d,n)
\end{sageblock}	

\end{homeworkProblem}


\begin{homeworkProblem}[Encriptando una palabra]

La palabra que se encriptar\'{a} es "PUCHICA"

\subsection{Codificaci\'{o}n}
La convenci\'{o}n que se uso es que para codificar la palabra a un n\'{u}mero para que pueda ser encriptado, usamos ASCIIs. En efecto, la palabra "ABC", ser\'{i}a codificada por el n\'{u}mero 656667.

Para codificar, usamos
\begin{sageblock}
def message_to_ascii_representation(message):
    ascii_rep = ""
    for character_index in range(0,len(message)):
        character = message[character_index]
        ascii_rep += str(ord(character))

    return int(ascii_rep)
\end{sageblock}

en donde codificando "PUCHICA", obtenemos el n\'{u}mero $m = 80856772736765$ .

\subsection{Decodificaci\'{o}n}
An\'{a}logamente a la codificaci\'{o}n, el proceso inverso estar\'{i}a implementado de la siguiente forma

\begin{sageblock}
def ascii_representation_to_message(ascii_rep):
    ascii_rep = str(ascii_rep)
    mssg = ""
    for hasta in range(0,len(ascii_rep),2):
        ord = int(ascii_rep[hasta:hasta+2])
        mssg += chr(ord)
    return mssg
\end{sageblock}


\subsection{Encriptando}

Para encriptar $m = 80856772736765 $ , necesitaremos nuestra llave p\'{u}blica $(4951760154835678088235319297, 17)$, que encriptando como describimos anteriormente, tenemos

\[ c = 4175976697833683195181896730\]

\subsection{Desencriptando}

Para desencriptar $c = 4175976697833683195181896730$ con nuestra llave privada, tenemos

\begin{sageblock}
sage: private_key
(2305843009213693951, 2147483647, 4077920125612805357425763753)
sage: desencriptado = get_rsa_number_decryption(4175976697833683195181896730,private_key)
sage: desencriptado
80856772736765
sage: ascii_representation_to_message(desencriptado)
'PUCHICA'
\end{sageblock}

Que es el mensaje que hab\'{i}amos encriptado inicialmente.
\end{homeworkProblem}

\end{spacing}
\end{document}

%%%%%%%%%%%%%%%%%%%%%%%%%%%%%%%%%%%%%%%%%%%%%%%%%%%%%%%%%%%%%
