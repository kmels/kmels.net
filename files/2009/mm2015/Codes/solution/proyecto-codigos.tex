\documentclass{article}
% Change "article" to "report" to get rid of page number on title page
\usepackage{amsmath,amsfonts,amsthm,amssymb}
\usepackage{setspace}
\usepackage{Tabbing}
\usepackage{fancyhdr}
\usepackage{lastpage}
\usepackage{extramarks}
\usepackage{chngpage}
\usepackage{soul,color}
\usepackage{sagetex}
\usepackage{graphicx,float,wrapfig}

% In case you need to adjust margins:
\topmargin=-0.45in      %
\evensidemargin=0in     %
\oddsidemargin=0in      %
\textwidth=6.5in        %
\textheight=9.0in       %
\headsep=0.25in         %

% Homework Specific Information
\newcommand{\hmwkTitle}{Proyecto C\'{o}digos y Grupos}
\newcommand{\hmwkDueDate}{Martes,\ Dic\ 1,\ 2009}
\newcommand{\hmwkClass}{MM2015}
\newcommand{\hmwkClassTime}{}
\newcommand{\hmwkClassInstructor}{H. Villafuerte}
\newcommand{\hmwkAuthorName}{Carlos E. L\'{o}pez Camey}

% Setup the header and footer
\pagestyle{fancy}                                                       %
\lhead{\hmwkAuthorName}                                                 %
\chead{\hmwkClass\ (\hmwkClassInstructor\ \hmwkClassTime): \hmwkTitle}  %
\rhead{\firstxmark}                                                     %
\lfoot{\lastxmark}                                                      %
\cfoot{}                                                                %
\rfoot{Page\ \thepage\ of\ \pageref{LastPage}}                          %
\renewcommand\headrulewidth{0.4pt}                                      %
\renewcommand\footrulewidth{0.4pt}                                      %

% This is used to trace down (pin point) problems
% in latexing a document:
%\tracingall

%%%%%%%%%%%%%%%%%%%%%%%%%%%%%%%%%%%%%%%%%%%%%%%%%%%%%%%%%%%%%
% Some tools
%\newcommand{\enterProblemHeader}[1]{\nobreak\extramarks{#1}{#1 continued on next page\ldots}\nobreak%
   %                                 \nobreak\extramarks{#1 (continued)}{#1 continued on next page\ldots}\nobreak}%
\newcommand{\exitProblemHeader}[1]{\nobreak\extramarks{#1 (continued)}{#1 continued on next page\ldots}\nobreak%
                                   \nobreak\extramarks{#1}{}\nobreak}%

\newlength{\labelLength}
\newcommand{\labelAnswer}[2]
  {\settowidth{\labelLength}{#1}%
   \addtolength{\labelLength}{0.25in}%
   \changetext{}{-\labelLength}{}{}{}%
   \noindent\fbox{\begin{minipage}[c]{\columnwidth}#2\end{minipage}}%
   \marginpar{\fbox{#1}}%

   % We put the blank space above in order to make sure this
   % \marginpar gets correctly placed.
   \changetext{}{+\labelLength}{}{}{}}%

\setcounter{secnumdepth}{0}
\newcommand{\homeworkProblemName}{}%
\newcounter{homeworkProblemCounter}%
\newenvironment{homeworkProblem}[1][Problem \arabic{homeworkProblemCounter}]%
  {\stepcounter{homeworkProblemCounter}%
   \renewcommand{\homeworkProblemName}{#1}%
   \section{\homeworkProblemName}%
   %\enterProblemHeader{\homeworkProblemName}
   }%
  %{\exitProblemHeader{\homeworkProblemName}}%

\newcommand{\problemAnswer}[1]
  {\noindent\fbox{\begin{minipage}[c]{\columnwidth}#1\end{minipage}}}%

\newcommand{\problemLAnswer}[1]
  {\labelAnswer{\homeworkProblemName}{#1}}

\newcommand{\homeworkSectionName}{}%
\newlength{\homeworkSectionLabelLength}{}%
\newenvironment{homeworkSection}[1]%
  {% We put this space here to make sure we're not connected to the above.
   % Otherwise the changetext can do funny things to the other margin

   \renewcommand{\homeworkSectionName}{#1}%
   \settowidth{\homeworkSectionLabelLength}{\homeworkSectionName}%
   \addtolength{\homeworkSectionLabelLength}{0.25in}%
   \changetext{}{-\homeworkSectionLabelLength}{}{}{}%
   \subsection{\homeworkSectionName}%
   \enterProblemHeader{\homeworkProblemName\ [\homeworkSectionName]}}%
  {\enterProblemHeader{\homeworkProblemName}%

   % We put the blank space above in order to make sure this margin
   % change doesn't happen too soon (otherwise \sectionAnswer's can
   % get ugly about their \marginpar placement.
   \changetext{}{+\homeworkSectionLabelLength}{}{}{}}%

\newcommand{\sectionAnswer}[1]
  {% We put this space here to make sure we're disconnected from the previous
   % passage

   \noindent\fbox{\begin{minipage}[c]{\columnwidth}#1\end{minipage}}%
   \enterProblemHeader{\homeworkProblemName}\exitProblemHeader{\homeworkProblemName}%
   \marginpar{\fbox{\homeworkSectionName}}%

   % We put the blank space above in order to make sure this
   % \marginpar gets correctly placed.
   }%

%%%%%%%%%%%%%%%%%%%%%%%%%%%%%%%%%%%%%%%%%%%%%%%%%%%%%%%%%%%%%


%%%%%%%%%%%%%%%%%%%%%%%%%%%%%%%%%%%%%%%%%%%%%%%%%%%%%%%%%%%%%
% Make title
\title{\textmd{\textbf{\hmwkClass:\ \hmwkTitle}}\\\normalsize\vspace{0.1in}\small{Para entregar\ el\ \hmwkDueDate}\\\vspace{0.1in}\large{\textit{\hmwkClassInstructor\ \hmwkClassTime}}}
\date{}
\author{\textbf{\hmwkAuthorName}}
%%%%%%%%%%%%%%%%%%%%%%%%%%%%%%%%%%%%%%%%%%%%%%%%%%%%%%%%%%%%%

\begin{document}
\begin{spacing}{1.1}
\maketitle
% Uncomment the \tableofcontents and \newpage lines to get a Contents page
% Uncomment the \setcounter line as well if you do NOT want subsections
%       listed in Contents
%\setcounter{tocdepth}{1}
%\tableofcontents


%\newpage

% When problems are long, it may be desirable to put a \newpage or a
% \clearpage before each homeworkProblem environment
\begin{homeworkProblem}[Grupos. ]
\subsection{(a) Seleccionando un grupo finito:}
Utilizemos $S_n$

\subsection{(b) Seleccionemos un entero n que determinar\'{a} el orden del grupo elegido}

\begin{sageblock}
sage: n = round(random()*10)
sage: n
9.0
sage: G = SymmetricGroup(9)
sage: G.order()
362880
\end{sageblock}

Vemos que el orden de G es de 362880.

\subsection{(c) Obtengamos un listado $H$ de subgrupos de $G$}

\begin{sageblock}
sage: H = G.normal_subgroups()
sage: H
[Permutation Group with generators [()],
 Permutation Group with generators [(2,3)(4,5), (2,4,3), (2,4,6,5,3), (2,5,3)(4,7,6), (2,8,7,4,5,6,3), (2,9,8,5,6,4,3), (1,3,2)],
 Permutation Group with generators [(1,2), (1,2,3,4,5,6,7,8,9)]]
\end{sageblock}

\subsection{(d) Orden de los subgrupos $H$ }

\begin{sageblock}
sage: [subgrupo.order() for subgrupo in H]
[1, 181440, 362880]
\end{sageblock}

Vemos que el \'{u}ltimo subgrupo del listado $H$ tiene el mismo orden que el grupo inicial $G$.

\begin{sageblock}
sage: G.order() == H[2].order()
True
\end{sageblock}

\subsection{(e) Repitamos algunas veces}

Hagamos lo siguiente en sage
\begin{sageblock}
Grupo = SymmetricGroup(7);


def verificar(G,H):
    return G.order() == H[len(H)-1].order()

for n in range(1,20):
    G = SymmetricGroup(n)
    print verificar(G,G.normal_subgroups())
\end{sageblock}

Vemos como respuesta
\begin{sageblock}
True
True
True
True
True
True
True
True
True
True
True
True
True
True
True
True
True
True
True
\end{sageblock}

Lo cual, demuestra nuestra congetura para todos los grupos $S_1\dots S_{20}$ aunque no para todos.

\end{homeworkProblem}


\begin{homeworkProblem}[C\'{o}digos]

\subsection{(a) Escojamos un mensaje $M$}

Escojamos "Superfragilisticoespiralidoso"
\begin{sageblock}
M = "Superfragilisticoespiralidoso"
\end{sageblock}

\subsection{(b) Codifiquemos $M$ empleando Hamming(7,4) sobre un canal que introduce 1-bit de error}

\begin{sageblock}
sage: C = HammingCode(3,GF(2))
sage: G = C.gen_mat()
sage: H = C.check_mat()
sage: Mv = str2vec(M,4)
sage: Mv = [c*G for c in Mv]
sage: Mv = [c + error_canal(7,1) for c in Mv]
sage: Mv = [C.decode(c) for c in Mv]
sage: Mv = [G.solve_left(c).list() for c in Mv]
sage: Mhat = vec2str(Mv)
sage: print M, '->', Mhat
Superfragilisticoespiralidoso -> Superfragilisticoespiralidoso
sage: print M==Mhat
True
\end{sageblock}

Vemos que el mensaje $M$ se pudo recuperar.

\subsection{(c) Codifiquemoslo ahora sobre un canal con 2-bits de error}

\begin{sageblock}
sage: C = HammingCode(3,GF(2))
sage: G = C.gen_mat()
sage: H = C.check_mat()
sage: Mv = str2vec(M,4)
sage: Mv = [c*G for c in Mv]
sage: Mv = [c + error_canal(7,1) for c in Mv]
sage: Mv = [C.decode(c) for c in Mv]
sage: Mv = [G.solve_left(c).list() for c in Mv]
sage: Mhat = vec2str(Mv)
sage: print M==Mhat
False
\end{sageblock}

Vemos que el mensaje $M$ NO se pudo recuperar.
\end{homeworkProblem}


\end{spacing}
\end{document}

%%%%%%%%%%%%%%%%%%%%%%%%%%%%%%%%%%%%%%%%%%%%%%%%%%%%%%%%%%%%%
