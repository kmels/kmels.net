\documentclass[10pt]{article}

\usepackage{calc}
  \usepackage{verbatim} 

\reversemarginpar
\usepackage[paper=letterpaper,
            marginparwidth=1.2in,     % Length of section titles
            marginparsep=.05in,       % Space between titles and text
            margin=1in,               % 1 inch margins
            includemp]{geometry}

\setlength{\parindent}{0in}
\usepackage{paralist}
\usepackage{fancyhdr,lastpage}
\pagestyle{fancy}
\fancyhf{}\renewcommand{\headrulewidth}{0pt}
\fancyfootoffset{\marginparsep+\marginparwidth}
\newlength{\footpageshift}
\setlength{\footpageshift}
          {0.5\textwidth+0.5\marginparsep+0.5\marginparwidth-2in}
\lfoot{\hspace{\footpageshift}%w
       \parbox{4in}{\, \hfill %
                   % \arabic{page} of \protect\pageref*{LastPage} % +LP	         
                               \arabic{page}                               % -LP
                    \hfill \,}}

% Finally, give us PDF bookmarks
\usepackage{color,hyperref}
\usepackage[latin1]{inputenc} 

\definecolor{darkblue}{rgb}{0.0,0.0,0.3}
\hypersetup{colorlinks,breaklinks,
            linkcolor=darkblue,urlcolor=darkblue,
            anchorcolor=darkblue,citecolor=darkblue}

\newcommand{\makeheading}[1]%
        {\hspace*{-\marginparsep minus \marginparwidth}%
         \begin{minipage}[t]{\textwidth+\marginparwidth+\marginparsep}%
                {\large \bfseries #1}\\[-0.15\baselineskip]%
                 \rule{\columnwidth}{1pt}%
         \end{minipage}}

\renewcommand{\section}[2]%
        {\pagebreak[2]\vspace{1.3\baselineskip}%
         \phantomsection\addcontentsline{toc}{section}{#1}%
         \hspace{0in}%
         \marginpar{
         \raggedright \scshape #1}#2}

\newenvironment{outerlist}[1][\enskip\textbullet]%
        {\begin{itemize}[#1]}{\end{itemize}%
         \vspace{-.6\baselineskip}}

\newenvironment{lonelist}[1][\enskip\textbullet]%
        {\vspace{-\baselineskip}\begin{list}{#1}{%
        \setlength{\partopsep}{0pt}%
        \setlength{\topsep}{0pt}}}
        {\end{list}\vspace{-.6\baselineskip}}

\newenvironment{innerlist}[1][\enskip\textbullet]%
        {\begin{compactitem}[#1]}{\end{compactitem}}

\newcommand{\blankline}{\quad\pagebreak[2]}

\begin{document}
\makeheading{Carlos Eduardo L\'{o}pez Camey}

\section{Informaci\'{o}n de contacto}
%
% NOTE: Mind where the & separators and \\ breaks are in the following
%       table.
%
% ALSO: \rcollength is the width of the right column of the table 
%       (adjust it to your liking; default is 1.85in).
%
\newlength{\rcollength}\setlength{\rcollength}{1.85in}%
%
\begin{tabular}[t]{@{}p{\textwidth-\rcollength}p{\rcollength}}

Ciudad de Guatemala, Guatemala  & \textit{M\'{o}vil:} (+502) 40400799 \\
%           & \textit{Work:} (+502) 2258-6028 \\
           & \textit{E-mail:} \href{mailto:c.lopez@kmels.net}{c.lopez@kmels.net} \\
	 & \href{http://www.kmels.net/}{http://www.kmels.net} \\

%\textit{Mobile:} (+502) 4473-2908 & \\ \href{http://kmels.net/}{http://kmels.net}
%\textit{Work:} (+502 2258-6028 & Guatemala City, Guatemala
\end{tabular}

\section{Fecha y lugar de nacimiento}
%
05/12/1989, Guatemala

%\section{Objective}
%
%Theoretical Computer Science, 
%Computer Science Internship

\section{Educaci\'{o}n}
\href{http://www.uni-freiburg.de/}{\textbf{Universidad de Friburgo}}, 
Friburgo, Alemania
\begin{outerlist}

\item[] Dos semestres en el programa de Ciencias de la Computaci\'{o}n (Oct 2012-Jul 2013)
\end{outerlist}

\blankline


%
\href{http://www.uvg.edu.gt/}{\textbf{Universidad Del Valle de Guatemala}}, 
Ciudad de Guatemala, Guatemala
\begin{outerlist}

\item[] 
       Tres a\~{n}os y medio en el programa B.Sc. de Ciencias de la Computaci\'{o}n (Ene 2008-Dic 2011) 
       \begin{innerlist}
       \item Presidente Interino de la Asociaci\'{o}n de Estudiantes de Ciencias de la Computaci\'{o}n, 2009
       \item Auxiliar de Profesor, Algoritmos y Estructuras de Datos, 2010 y 2011
        \end{innerlist}
\end{outerlist}

\blankline

\href{http://www.suizoamericano.edu.gt/}{\textbf{Colegio Suizo Americano}}, 
Ciudad de Guatemala, Guatemala
\begin{outerlist}
\item[] Bachillerato en Computaci\'{o}n
        (2006-2007)
\end{outerlist}

\blankline

\section{Becas}
\href{http://www.uu.nl/}{\textbf{Universiteit Utrecht}}, Utrecht, Holanda

\begin{outerlist}
\item[] \href{http://www.utrechtsummerschool.nl/index.php?type=courses&code=H9}{Programaci\'{o}n Funcional Aplicada en Haskell} (2012)
        \hfill Beca completa (835 euros)
\end{outerlist}

\blankline

\href{http://www.sigplan.org/}{\textbf{ACM SIGPLAN-SIGACT}}, Roma, Italia, 2013 

\begin{outerlist}
\item[] \href{http://popl.mpi-sws.org/2013/}{40vo. Simposio sobre Principios de Lenguajes de Programaci\'{o}n} (2013)
        \hfill Registro y viaticos (950 d\'{o}lares)
\end{outerlist}

\blankline

\section{Experiencia Profesional}
\href{http://proglang.informatik.uni-freiburg.de/}{\textbf{Grupo de Lenguajes de Programaci\'{o}n}}, Universidad de Friburgo
\begin{outerlist}
\item[] \textit{Asistente de Investigaci\'{o}n}%
        \hfill \textbf{Febrero 2013 - Julio 2013}
        \begin{innerlist}
        \item \href{http://github.com/kmels/dart-haskell}{\textbf{dart-haskell}} (\textit{trabajo en progreso}), un programa que lee, analiza y simula programas escritos en Haskell recurrentemente para poder encontrar errores o argumentos que lo hacen fallar. 
          \item Trabajo en conjunto con Prof. Peter Thiemann basado en \href{http://doi.acm.org/10.1145/1065010.1065036}{Godefroid, Patrice and Klarlund, Nils and Sen, Koushik, DART: Directed Automated Random Testing}, PLDI2005, 2005}
        \end{innerlist}
\end{outerlist}

\blankline

% Digitalgeko
\href{http://www.digitalgeko.com/}{\textbf{Digitalgeko}}, 
\begin{outerlist}
\item[] \textit{Desarrollador Web}%
        \hfill \textbf{Agosto 2010 - Enero 2011}
        \begin{innerlist}
        \item Participaci\'{o}n en el desarrollo de dos sistemas para el gobierno de Timor-Leste en un equipo de 5 personas. Tecnolog\'{i}as Java fueron usadas (Spring Framework, Hibernate, Stripes Framework). 
          \begin{innerlist}
              \item \href{http://www.budgettransparency.gov.tl/public/index?&lang=en}{\textbf{Timor-Leste Budget Transparency Portal}},
              \item \href{http://www.eprocurement.gov.tl/public/indexeprtl?&lang=en}{\textbf{Timor-Leste eProcurement Portal}}
          \end{innerlist}
        \item Implementaci\'{o}n de un sitio web basado en el manejador de contenidos TYPO3.
        \end{innerlist}
\end{outerlist}

\blankline

{\textbf{Elex de Guatemala S.A.}}
\begin{outerlist}
\item[] \textit{Freelance, } \hfill \textbf{Enero 2011 - Mayo 2011}
        \begin{innerlist}
        \item Redise\~{n}o de la base de datos para implementar la interface de administraci\'{o}n en la p\'{a}gina web en \href{http://elexsa.com}{elexsa.com}
        \item Implementaci\'{o}n del sitio web para m\'{o}viles en \href{http://elexsa.com/movil}{elexsa.com/movil}
        \end{innerlist}
\end{outerlist}

\blankline

\href{http://www.tecnomaya.net/}{\textbf{Tecnomaya S.A. \& la Casa de la Bombilla S.A.}}, 
Ciudad de Guatemala, Guatemala
\begin{outerlist}

\item[] \textit{Desarrollador Web / Administrador de Sistemas.}%
        \hfill \textbf{Enero 2000 - Abril 2001}
\begin{innerlist}
\item Desarrollo de un sistema de control de prorrateos.
\item El sistema en \textit{Intranet} basado en SugarCRM y la plataforma Xoops.
\item http://www.tecnomaya.net
\item http://www.cabsagt.com
\end{innerlist}
\end{outerlist}

\blankline

\section{Contribuciones a proyectos \it{Open source}}

\textbf{XMonad}, trabajo en XMonad.Prompt e implementaci\'{o}n de XMonad.Actions.Launcher; una consola escrita en Haskell que funciona sobre el manejador de gr\'{a}ficos X11.

\blankline

\section{T\'{e}cnico} Lenguajes de programaci\'{o}n: Java (avanzado, 2 a\~{n}os d experiencia), Haskell (avanzado, 2 a\~{n}os de experiencia), Scala (intermedio-avanzado, 1 a\~{n}o de experiencia), PHP (intermedio), Python, Pascal (Delphi), MySQL, Applescript.

\blankline

Sistemas de Control de Versiones: SVN, Mercurial.

\blankline

Sistemas operativos: Mac OS X, Linux Ubuntu, Windows XP/Vista y CentOS.

\blankline

Otros: \LaTeX{}, PGF (TikZ), Sage, Wordpress, gnuplot, Emacs, Apache Ant, Hibernate framework 

\blankline

\section{Intereses}
Generales: Ciencia de la computaci\'{o}n te\'{o}rica, productos innovadores y Software libre.

\blankline

Espec\'{i}ficos: Lenguajes y programaci\'{o}n funcional, teor\'{i}a de tipos, compiladores.

\section{Membres\'{i}as}
\textbf{Association of Computing Machinery} (ACM)

\blankline

\textbf{Special Interest Group on Programming Languages} (SIGPLAN)

\blankline

\textbf{Universidad del Valle de Guatemala, Asociaci\'{o}n de Estudiantes de Ciencias de la Computaci\'{o}n} (AECC)

\section{Idiomas}
Espa\~{n}ol (Nativo), Ingl\'{e}s (flu\'{i}do), Alem\'{a}n (flu\'{i}do)

Certificados oficiales:
\begin{outerlist}  
  \item \textbf{DSH-2}, Centro de Lenguajes, Universidad de Friburgo. \hfill \textbf{2012}
  \item \textbf{TOEFL}, Avanzado en escritura, audici\'{o}n, escritura e intermedio en elocuencia. \hfill \textbf{2012}
  \item \textbf{ELASH}, Colegio Suizo Americano    \hfill \textbf{2007}
  \item  \textbf{Embassy Certificate of English}, Embasy CES \hfill \textbf{2007}
\end{outerlist}

\section{Cursos tomados} \textbf{Applescript with Sal Soghoian}, MacWorld Expo 2009, \textit{San Francisco, CA.} \hfill \textbf{Jan 2009}

%\textbf{http://archive.kmels.net/courses} y \textbf{http://uvg.kmels.net/courses.html}, Cursos que estoy tomando/he tomado en la Universidad, \textit{Guatemala}

%\blankline

%\textbf{New Media Artists}, MacWorld Expo 2009, \textit{San Francisco, CA.}%
%\hfill \textbf{Jan 2009}


\blankline

\textbf{Embassy Certificate of English}, Embasy CES,
\textit{Hastings, UK}  \hfill \textbf{Nov 2007}

\blankline

\textbf{Presentaciones Juveniles, Dale Carnegie Training}, Dale Carnegie \& Associates, Inc., \textit{Guatemala City, Guatemala}  \hfill \textbf{March 2004}

\blankline

\textbf{Dale Carnegie Course for Teenagers}, Dale Carnegie \& Associates, Inc., \textit{Guatemala City, Guatemala}  \hfill \textbf{Dec 2003}

\blankline


\blankline

%\section {Referencias} \textbf{H\'{e}ctor Villafuerte (hfvillafuerte@uvg.edu.gt, +502 5305-3860)},Profesor,Departmento de Matem\'{a}tica, Universidad del Valle de Guatemala.

%\textbf{Douglas Barrios (douglas.barrios1@gmail.com, +502 50125586)},Profesor,Departamento de Ciencias de la Computaci\'{o}n, Universidad del Valle de Guatemala.
 
%\textbf{Luis Furl\'{a}n (furlan@uvg.edu.gt, +502 2368 8566)},Director,Departmento de Ciencias de la Computaci\'{o}n, Universidad del Valle de Guatemala. 

\end{document}

%%%%%%%%%%%%%%%%%%%%%%%%%% End CV Document %%%%%%%%%%%%%%%%%%%%%%%%%%%%%
